\documentclass[11pt]{article}
\usepackage{fullpage}
\usepackage{framed}
\usepackage[numbers]{natbib}
\usepackage{dsfont}
\usepackage{latexsym}
\usepackage{amsmath}
\usepackage{amsthm}
\usepackage{amssymb}
\usepackage{amsfonts}
\usepackage{mathrsfs}
\usepackage{aliascnt}
\usepackage{bold-extra}

% \usepackage{pstricks}
% \usepackage{pst-all}
% \usepackage{pstricks-add}
% \usepackage{pst-plot}
\usepackage{graphicx}
% \usepackage{subfig}
\usepackage[margin=1in]{geometry}
% \usepackage{enumerate}
\usepackage[pdfpagelabels,pdfpagemode=None]{hyperref}
% \usepackage{algorithmic}
\usepackage{algorithm}
\usepackage{algpseudocode}
\usepackage[table,xcdraw]{xcolor}

\usepackage{booktabs}
\usepackage{multirow}

% \usepackage{longtable}
\usepackage{fancyvrb}

\usepackage{lscape}


\usepackage{subcaption}
\delimiterfactor=1100

\setlength{\delimitershortfall}{0pt}

\usepackage{notation}

\usepackage{array}
\newcolumntype{L}[1]{>{\raggedright\let\newline\\\arraybackslash\hspace{0pt}}m{#1}}
\newcolumntype{C}[1]{>{\centering\let\newline\\\arraybackslash\hspace{0pt}}m{#1}}
\newcolumntype{R}[1]{>{\raggedleft\let\newline\\\arraybackslash\hspace{0pt}}m{#1}}

\title{\textbf{\textsc{Genetic Algorithm and Graph Partitioning}}}
\graphicspath{ {images/} }

\date{
    \today
}

\author{
  Zhenpeng Wu\\
  28125152
}

\begin{document}

\maketitle

\section{\textsc{Project Overview}}

Given a graph \(G = (V, E)\) on \(n\) vertices, where \(V\) is the set of vertices and \(E\) is the 
set of edges, a balanced \textit{two-way partition} is a partitioning of the vertex set \(V\) into 
\textit{two} disjoint subsets where the difference of cardinalities between them is at most one. The
graph bisection problem is a well-known \(\classnp\)-hard problem. There are a number of heuristics 
algorithm that seems to work well but have no performance guarantee are used, and the most popular 
one is \textit{the Kernighan-Lin Algorithm (KL)}~\cite{6771089} or \textit{the Fiduccia-Mattheyses Algorithm (FM)}~\cite{1585498}. 
In addition, Bui~\textit{et.\,al.}~\cite{508322} also proposed a \textit{Genetic Algorithm (GA)} for 
the graph partitioning problem.

In this project, I reimplement the KL / FM Algorithm and Bui's Genetic Algorithm and analyze their 
performance. All of the source code and the report are hosted on GitHub (\url{https://github.com/ZhenpengWu/Genetic-Partitioning}). 
A \texttt{README} file containing instructions on how to run the program is also included in the 
repository.

\section{\textsc{Kernighan-Lin / Fiduccia-Mattheyses Algorithm}}

\subsection{Overview of the Algorithm}

The Kernighan-Lin / Fiduccia-Mattheyses algorithm for finding a bisection of a graph is a local 
optimization algorithm that improves upon a given initial bisection by swapping equal-sized subsets 
of the bisection to create a new bisection. This process is repeated on the new bisection either a 
fixed number of times or until no improvement can be obtained. Fig.~\ref{alg:kl} shows the structure 
of the KL / FM algorithm.

\begin{figure}[H]
	\small
	\begin{Verbatim}[tabsize=4,xleftmargin=5mm,numbers=left]
		for a small number of passes {
			unlock all nodes;
			while some nodes are unlocked {
				calculate all gains;
				choose node with highest gain 
					whose movement would not cause an imbalance;
				move node to other block and lock it;
			}
			choose the best cut seen in this pass;
		}
	\end{Verbatim}
	\caption{The Kernighan-Lin / Fiduccia-Mattheyses Algorithm}
	\label{alg:kl}
\end{figure}


\subsection{Handling of the Multi-Sink Nets}

To handle the multi-sink nets, my implementation follows the implementation of the algorithm 
described in~\cite{1585498}. First, a \textit{net's distribution} as a tuple is defined for each net,  
where the first number is the number of nodes in the first block, and the second number is the 
number of nodes in the second block. Also, the authors defined a \textit{critical net} as a net 
that has a node that if moved would change the net's cut state. Also, based on the the authors' 
observation, a net is critical if, in its distribution, at least one of the numbers in the tuple 
is equal to \(0\) or \(1\).

\subsection{Efficiency}

With the efficiency consideration, my implementation follows the implementation of the algorithm 
described in~\cite{1585498} as well. Specifically, the authors defined \(P\) as the total number 
of pins in the circuit. The runtime of the original work is worse than \(O(P^2)\) per iteration, 
but the runtime of their implementation is reduced to \(O(P)\) per iteration.

The main work of an iteration involves selecting a base node with the highest gain value, moving 
it to another block, and updating the gains of the neighbor node. Therefore, first, in order to 
select a base node efficiently, a bucket array of possible gain values is used, and it ranges from 
\texttt{-pmax} to \texttt{pmax}, where \texttt{pmax} is the maximum number of pins across all nodes. 
Each bucket contains a list of nodes that have the same gain value as its index. The use of such a 
structure allows selecting a base node in constant time. Second, the authors defined routines for 
efficiently updating the gains of all neighbor nodes of the base cell. Essentially the various 
updates are turned into a series of gain increments and decrements. Therefore, to combine the fact 
that only critical nets need to have their node gains updated, it allows implementing this step 
in linear time.

\section{\textsc{Genetic Algorithm For Graph Partitioning}}

\subsection{Overview of the Algorithm}

A genetic algorithm starts with a set of initial solutions (\textit{chromosomes}), called a 
\textit{population}. In my implementation, the population size \(p\) is set to \(50\). This 
population then evolves into different populations for several iterations. In the end, the algorithm
returns the best member of the population as the solution to the problem. For each iteration, the 
evolution process proceeds as follows. Two members of the population are chosen as parents based on
some probability distribution. These two members are then combined through a \textit{crossover} 
operator to produce offspring. Then, this offspring is then modified by a \textit{mutation} operator
to introduce unexplored search space to the population, enhancing the diversity of the population. 
The offspring is tested to see if it is suitable for the population. In addition, the genetic 
algorithm generates only one offspring per generation. If it is, a \textit{replacement} scheme is 
used to select a member of the population and replace it with the new offspring. Then, the evolution 
process is repeated until a certain condition is met. Otherwise, a \textit{local improvement} 
heuristic, typically after mutation, it would be called \textit{hybrid} GA. Fig.~\ref{alg:ga} shows 
the structure of the genetic algorithm for graph bisection.

\begin{figure}[H]
	\small
	\begin{Verbatim}[tabsize=4,xleftmargin=5mm,numbers=left]
		create initial population of fixed size p;
		do {
				choose parent1 and parent2 from population;
				offspring = crossover(parent1, parent2);
				mutation(offspring);
				local-improvement(offspring);
				replace(population, offspring);
			} until (stopping condition);
		report the best answer;
	\end{Verbatim}	
	\caption{The Genetic Algorithm for Graph Bisection}
	\label{alg:ga}
\end{figure}


\subsection{Parents Selection}

In order to select a parent proportionally (\textit{proportional selection}), a fitness value is
calculated for each solution in the population from its cut size. The fitness value \(F_i\) of a 
solution \(i\) is calculated as follows:

\begin{align*}
	F_i = (C_w - C_i) + (C_w - C_b) / 3,
\end{align*}

\noindent
where \(C_w\) is the cut size of the worst solution in the population, \(C_b\) is the cut size of 
the best solution in the population, and \(C_i\) is the cut size of solution \(i\).


\subsection{The Crossover and Mutation Operators}

A crossover operator creates a new offspring chromosome by combining parts of the two parent 
chromosomes. My implementation employs the 5 cut points crossover, a cut point divides the 
chromosome into two disjoint parts. 5 cut points are randomly selected and are the same on both
parent chromosomes, and they divide the parent chromosome into 6 disjoints part. For the first 
crossover operator, let this be offspring 1, the even part of parent 1 is copied to the same 
location of offspring 1. Similarly, the odd part of parent 2 is copied to the same locations
of offspring 1. For the second crossover operator, let this be offspring 2, it is the same as
the first operation except that the complement value of the odd part of parent 2 is copied to 
the same locations of offspring 2. Fig.~\ref{fig:M1} shows an example of 5 cut points crossover.

\begin{figure}[H]
	\centering
	\resizebox{\linewidth}{!}{

		  \begin{tikzpicture}[-,>=stealth',shorten >=1pt,auto,node distance=2cm,semithick]
			
				\tikzstyle{every state}=[text=black]			
				\tikzstyle{every node}=[font=\fontsize{60}{0}\selectfont]			

				\node[red] (p1p1) {000};
				\node[right of=p1p1, blue] (p1p2) {110};
				\node[right of=p1p2, red] (p1p3) {0000};
				\node[right of=p1p3, blue] (p1p4) {101};
				\node[right of=p1p4, red] (p1p5) {11};
				\node[right of=p1p5, blue] (p1p6) {11011};

				\node[below of=p1p1,yshift=-1.5cm,xshift=-20em,red] (o1p1) {000};
				\node[below of=p1p2,yshift=-1.5cm,xshift=-20em,blue] (o1p2) {111};
				\node[below of=p1p3,yshift=-1.5cm,xshift=-20em,red] (o1p3) {0000};
				\node[below of=p1p4,yshift=-1.5cm,xshift=-20em,blue] (o1p4) {000};
				\node[below of=p1p5,yshift=-1.5cm,xshift=-20em,red] (o1p5) {11};
				\node[below of=p1p6,yshift=-1.5cm,xshift=-20em,blue] (o1p6) {00000};

				\node[below of=p1p1,yshift=-1.5cm,xshift=20em,red] (o2p1) {000};
				\node[below of=p1p2,yshift=-1.5cm,xshift=20em,blue] (o2p2) {000};
				\node[below of=p1p3,yshift=-1.5cm,xshift=20em,red] (o2p3) {0000};
				\node[below of=p1p4,yshift=-1.5cm,xshift=20em,blue] (o2p4) {111};
				\node[below of=p1p5,yshift=-1.5cm,xshift=20em,red] (o2p5) {11};
				\node[below of=p1p6,yshift=-1.5cm,xshift=20em,blue] (o2p6) {11111};
				
				\node[below of=p1p1,yshift=-5cm,red] (p2p1) {011};
				\node[below of=p1p2,yshift=-5cm,blue] (p2p2) {111};
				\node[below of=p1p3,yshift=-5cm,red] (p2p3) {1011};
				\node[below of=p1p4,yshift=-5cm,blue] (p2p4) {000};
				\node[below of=p1p5,yshift=-5cm,red] (p2p5) {11};
				\node[below of=p1p6,yshift=-5cm,blue] (p2p6) {00000};

				\node[draw,dotted,fit=(p1p1) (p1p6)] (p1box) {};
				\node[draw,dotted,fit=(o1p1) (o1p6)] (o1box) {};
				\node[draw,dotted,fit=(p2p1) (p2p6)] (p2box) {};
				\node[draw,dotted,fit=(o2p1) (o2p6)] (o2box) {};

				\node[left of=p1box,xshift=-20em] (p1) {\textbf{parent1}};
				\node[right of=p2box, xshift=20em] (p2) {\textbf{parent2}};
				\node[left of=o1box, xshift=-20em] (o1) {\textbf{offspring1}};
				\node[right of=o2box, xshift=20em] (o2) {\textbf{offspring2}};

				\draw
				(p1p1.south) edge[-{Latex[length=3mm]}] node{} (o1p1.north)
				(p1p1.south) edge[-{Latex[length=3mm]}] node{} (o2p1.north)
				(p2p2.north) edge[-{Latex[length=3mm]}] node{} (o1p2.south)
				(p2p2.north) edge[-{Latex[length=3mm]}] node{} (o2p2.south)

				(p1p3.south) edge[-{Latex[length=3mm]}] node{} (o1p3.north)
				(p1p3.south) edge[-{Latex[length=3mm]}] node{} (o2p3.north)
				(p2p4.north) edge[-{Latex[length=3mm]}] node{} (o1p4.south)
				(p2p4.north) edge[-{Latex[length=3mm]}] node{} (o2p4.south)

				(p1p5.south) edge[-{Latex[length=3mm]}] node{} (o1p5.north)
				(p1p5.south) edge[-{Latex[length=3mm]}] node{} (o2p5.north)
				(p2p6.north) edge[-{Latex[length=3mm]}] node{} (o1p6.south)
				(p2p6.north) edge[-{Latex[length=3mm]}] node{} (o2p6.south)
				;

				\draw 
				(p1.east) edge[-{Latex[length=3mm]}] node{} (p1box.west)
				(o1.east) edge[-{Latex[length=3mm]}] node{} (o1box.west)
				(p2.west) edge[-{Latex[length=3mm]}] node{} (p2box.east)
				(o2.west) edge[-{Latex[length=3mm]}] node{} (o2box.east);

		  \end{tikzpicture}
	}
	\caption{Crossover operator with five cut points} \label{fig:M1}
\end{figure}

A mutation operator as follows. \(m\) positions on the chromosome are selected at random and their
values are reversed, where \(m\) is a uniform random integer variable on the interval \([0, n/100]\). 
In addition, to maintain the balanced condition, a random point on the chromosome is selected and 
change the number 1s to 0s (or 0s to 1s) starting at that point on to the right until the chromosome 
is balanced.


\subsection{Local Improvement}

After crossover and mutation, a local improvement is applied to the offspring. With local improvement, 
GAs usually converge in considerably fewer iterations. In Bui's GA, a pass of KL algorithm is used 
to perform the local improvement with the consideration of runtime.

\subsection{Replacement Scheme}

After having generated a new offspring and trying to locally improve it, GA replaces a member of 
the population with the new offspring. A combined replacement scheme is used, when an offspring 
tries to first replace the more similar parent, measured in Hamming distance and if it fails, then 
it tries to replace the other parent. But if the offspring is worse than both parents, the most 
inferior member of the population is replaced.

\subsection{Stoppint Criterion}

Because the combined replacement scheme is used, the stopping criterion is to stop when 80\% of 
the population is occupied by solutions with the same quality, whose chromosomes are not necessarily 
the same.


\section{\textsc{Experimental Results}}

I implemented the algorithms in Python. In order to evaluate the implemented algorithms, the 
experiments were conducted on the benchmarks provided in the assignments, and the results is shown 
in Table~\ref{table:result} and all graphic results could be found in the \href{https://github.com/ZhenpengWu/Genetic-Partitioning/tree/master/docs/images}{\texttt{images}} directory.

\begin{table}[H]
	\small
	\centering
	\begin{tabular}{ C{8em} |  C{5em} |  C{5em} | C{5em} |  C{5em} | C{5em}| C{5em} }
		\hline
		\multirow{2}{*}{\textbf{Benchmark}}
		& \multicolumn{2}{c|}{\textbf{Branch and Bound}}
		& \multicolumn{2}{c|}{\textbf{Kernighan-Lin}}
		& \multicolumn{2}{c}{\textbf{Genetic}} \\ \cline{2-7}
		& \textbf{Cost} & \textbf{Runtime}
		& \textbf{Cost} & \textbf{Runtime}
		& \textbf{Cost} & \textbf{Runtime} \\
		\hline
		\hline
		alu2             & N/A & N/A          & 30    & 4 s          & 23    & 50 s         \\
		\hline
		apex1            & N/A & N/A          & 140   & 5 s          & 107   & 312 s        \\
		\hline
		apex4            & N/A & N/A          & 155   & 8 s          & 144   & 567 s        \\
		\hline
		C880             & N/A & N/A          & 35    & 5 s          & 27    & 53 s         \\
		\hline
		cc               & 4   & 57 min       & 4     & 3 s          & 4     & 7 s          \\
		\hline
		cm82a            & 1   & \(\leq\) 1 s & 1     & 2 s          & 1     & 3 s          \\
		\hline
		cm138a           & 4   & 1 s          & 4     & 4 s          & 4     & 4 s          \\
		\hline
		cm150a           & 6   & 9 min        & 6     & 1 s          & 6     & 4 s          \\
		\hline
		cm151a           & 5   & 1 s          & 4     & 1 s          & 4     & 2 s          \\
		\hline
		cm162a           & 6   & 2 min        & 6     & 3 s          & 6     & 6 s          \\
		\hline
		con1             & 4   & \(\leq\) 1 s & 4     & 1 s          & 4     & 3 s          \\
		\hline
		cps              & N/A & N/A          & 115   & 6 s          & 92    & 447 s        \\
		\hline
		e64              & N/A & N/A          & 61    & 5 s          & 47    & 175 s        \\
		\hline
		paira            & N/A & N/A          & 62    & 6 s          & 2     & 230 s        \\
		\hline
		pairb            & N/A & N/A          & 59    & 6 s          & 2     & 135 s        \\
		\hline
		twocm            & N/A & N/A          & 1     & 4 s          & 1     & 11 s         \\
		\hline
		ugly8            & 8   & \(\leq\) 1 s & 8     & \(\leq\) 1 s & 8     & \(\leq\) 1 s \\
		\hline
		ugly16           & 16  & 2 s          & 16    & \(\leq\) 1 s & 16    & \(\leq\) 1 s \\
		\hline
		z4ml             & 3   & 2 s          & 3     & 2 s          & 3     & 4 s          \\
		\hline
		\hline
		\textbf{Average} & N/A & N/A          & 37.68 & 3.37 s       & 26.53 & 110.53 s     \\
		\hline
	\end{tabular}
	\caption{Table of Results}
	\label{table:result}
\end{table}

Based on the table above, we could observed that, first, as we know that the \textit{Branch and Bound Partitioning Algorithm} 
guarantee to find the optimal solution but the runtime is exponential, thus about half of benchmarks
are unsolvable. Second, to compare KL and GA, the minimum cut size found by GA is always better than
that of KL, however, the runtime of KL is much smaller than that of GA, because the number of iterations 
in KL is small and fixed, whereas the number of iterations in GA is dynamically (typically hundreds of 
iterations).


\begin{figure}[H]
	\centering
	\begin{subfigure}[b]{\textwidth}
		\centering
		\includegraphics[width=0.8\textwidth]{paira_kl.png}
		\caption{KL}
	\end{subfigure}
	\hfill
\end{figure}
\begin{figure}[H]\ContinuedFloat
	\centering
	\begin{subfigure}[b]{\textwidth}
		\centering
		\includegraphics[width=0.8\textwidth]{paira_genetic.png}
		\caption{Genetic}
	\end{subfigure}
	\caption{Experiment Result on \texttt{paira}}
\end{figure}

\noindent

\newpage


\bibliographystyle{unsrt}

\bibliography{sample.bib}

\nocite{*}

\end{document}


